\documentclass[11pt]{article}
\usepackage[a4paper, margin=1in]{geometry}
\usepackage{graphicx}
\usepackage{xcolor}
\usepackage{hyperref}
\usepackage{enumitem}
\usepackage{fontawesome}
\usepackage{tcolorbox}
\usepackage{fancyhdr}
\usepackage{tabularx}
\usepackage{booktabs}
\usepackage[romanian]{babel}
\usepackage[utf8]{inputenc}

% Define colors
\definecolor{primary}{RGB}{0, 123, 255}
\definecolor{secondary}{RGB}{108, 117, 125}
\definecolor{success}{RGB}{40, 167, 69}
\definecolor{dark}{RGB}{50, 54, 66}

% Set up hyperlinks
\hypersetup{
    colorlinks=true,
    linkcolor=primary,
    urlcolor=primary,
}

% Set up headers and footers
\pagestyle{fancy}
\fancyhf{}
\renewcommand{\headrulewidth}{0pt}
\fancyfoot[C]{\thepage}
\fancyhead[L]{\textcolor{primary}{SpaceApps Challenge 2025}}
\fancyhead[R]{\textcolor{primary}{4-5 Octombrie 2025}}

% Custom commands
\newcommand{\sectiontitle}[1]{
    \vspace{0.3cm}
    \noindent\textcolor{primary}{\Large\textbf{#1}}
    \vspace{0.2cm}
}

\newcommand{\eventitem}[2]{
    \noindent\textbf{#1} \hfill #2\\
}

\begin{document}

\begin{center}
    \vspace*{-1cm}
    \textcolor{primary}{\Huge\textbf{COMUNICAT DE PRESĂ}}\\
    \vspace{0.5cm}
    \textcolor{primary}{\Large\textbf{CRAIOVA GĂZDUIEȘTE A 5-A EDIȚIE A HACKATHONULUI GLOBAL NASA SPACEAPPS CHALLENGE 2025}}\\
    \vspace{0.3cm}
    \textcolor{secondary}{\large Cel mai mare hackathon anual global ajunge din nou la Craiova pe 4-5 octombrie 2025}\\
    \vspace{0.5cm}
    {\normalsize Craiova, 24 iunie 2025}\\
    \vspace{0.8cm}
\end{center}

\begin{center}
\includegraphics[width=0.6\textwidth]{/home/alin/Documents/Stoic/dev/websites/chenist.dev/chen.ist/chenist-gh/chenist/assets/logo/spaceappschallenge.png}
\end{center}

\begin{tcolorbox}[colback=primary!10, colframe=primary, title=\textbf{Prezentare Generală a Evenimentului}]
Organizatorii de la PR Pătrat și Chenist Academy anunță cu mândrie că orașul Craiova va găzdui din nou prestigiosul hackathon NASA SpaceApps Challenge în perioada 4-5 octombrie 2025. Acest eveniment marchează a 9-a ediție organizată de echipa locală, consolidând astfel poziția Craiovei pe harta globală a inovației tehnologice spațiale și continuând o tradiție de succes care a început în 2018.
\end{tcolorbox}

\sectiontitle{Despre SpaceApps Challenge}

SpaceApps Challenge este cel mai mare hackathon anual global, care angajează în fiecare an mii de solutionatori de probleme din întreaga lume. Evenimentul promovează rezolvarea colaborativă a problemelor cu scopul de a produce soluții open-source relevante pentru a aborda necesitățile globale aplicabile atât vieții de pe Pământ, cât și vieții din spațiu.

Timp de 48 de ore intensive, echipele multidisciplinare au acces la datele publice ale NASA și ale partenerilor săi din agențiile spațiale pentru a colabora în dezvoltarea de soluții inovatoare la provocări reale din lumea contemporană. Misiunea NASA Space Apps Challenge este de a stimula colaborarea, creativitatea și gândirea critică, de a încuraja interesul pentru explorarea Pământului și a spațiului, de a crește gradul de conștientizare globală asupra datelor NASA și de a dezvolta următoarea generație de oameni de știință, ingineri, specialiști în tehnologie și designeri.

De la lansarea sa în 2012, acest program inovator de incubare al NASA a cooptat peste 220.000 de persoane din peste 185 de țări și teritorii, fiind administrat de Divizia de Științe a Pământului din cadrul Direcției Misiunii Științifice de la sediul NASA din Washington, DC.

\sectiontitle{Program și Activități}

\textbf{Bootcamp-uri Pre-Hackathon:}
\begin{itemize}[leftmargin=*]
    \item 10 iulie 2025 - locația și orele vor fi anunțate
    \item 10 august 2025 - locația și orele vor fi anunțate
    \item 10 septembrie 2025 - locația și orele vor fi anunțate
\end{itemize}
Bootcamp-urile vor fi organizate la locațiile oficiale "Outpost" ale evenimentului.

\textbf{Programul Hackathonului:}\\

\textbf{Sâmbătă, 4 octombrie 2025:}
\begin{itemize}[leftmargin=*]
    \item Orientarea obligatorie pentru participanți
    \item Check-in, cina și configurarea conturilor software necesare
    \item Prezentarea provocărilor și a sponsorilor
    \item Formarea echipelor și pregătirea pentru weekendul de programare
\end{itemize}

\textbf{Duminică, 5 octombrie 2025:}
\begin{itemize}[leftmargin=*]
    \item 09:00 - Micul dejun și lansarea oficială a hackathonului
    \item Dezvoltarea proiectelor cu suport de la mentori și organizatori
    \item 12:00 - Termenul limită pentru predarea proiectelor
    \item După-amiaza - Demonstrațiile echipelor (transmise live la nivel global)
    \item 17:00 - Anunțarea câștigătorilor și premierea
\end{itemize}

\sectiontitle{Organizatori, Sponsori și Parteneri}

\textbf{Organizatori:}
\begin{itemize}[leftmargin=*]
    \item PR Pătrat
    \item Chenist Academy
\end{itemize}

\textbf{Sponsor Principal:}
\begin{itemize}[leftmargin=*]
    \item Chenist Cybersecurity
\end{itemize}

\textbf{Parteneri Strategici:}
\begin{itemize}[leftmargin=*]
    \item Universitatea din Craiova
    \item Școala Romano-Britanică
    \item Chenist Digital
    \item Cluster EduTIC
    \item ARTI
    \item Roofs.Center
    \item Society for Computing Technologies
\end{itemize}

\textbf{Parteneri Media:}
\begin{itemize}[leftmargin=*]
    \item Urmează să fie anunțați
\end{itemize}

\textbf{Locații "Avanpost" Oficiale:}
\begin{itemize}[leftmargin=*]
    \item Urmează să fie anunțate
\end{itemize}

\begin{tcolorbox}[colback=secondary!10, colframe=secondary, title=\textbf{O Tradiție de Excelență cu Impact Global}]
SpaceApps Challenge Craiova a devenit un eveniment de referință în peisajul tehnologic românesc din 2018, când s-a desfășurat prima ediție locală la Muzeul Olteniei. În 2018, evenimentul a atras 87 de participanți care au format 12 echipe, demonstrând interesul crescut al comunității locale pentru inovația spațială.

Succese remarcabile din edițiile anterioare:
\begin{itemize}
    \item În 2022, locul I a fost câștigat de Andrei Dumitru, un elev din clasa a XI-a de la Liceul Teoretic "Ionita Asan" din Caracal
    \item Ediția din 2019 a fost găzduită de Muzeul Olteniei și a atras aproximativ 100 de participanți, majoritatea cu profil tehnic
    \item Evenimentul a demonstrat capacitatea de a pune laolaltă oameni deschiși din toate domeniile, creând oportunități de networking și colaborare interdisciplinară
\end{itemize}

Organizatorul principal, Alin Mechenici, aduce o experiență vastă, fiind la a IX-a ediție de SpaceApps Challenge în 2025, după ce anterior a co-organizat edițiile din 2016 și 2017 în Mjardevi Science Park, Linköping, Suedia.
\end{tcolorbox}

\sectiontitle{Participare și Accesibilitate}

Participarea la hackathon este gratuită, acest lucru fiind posibil datorită generozității sponsorilor.

Evenimentul va avea un format hibrid, permițând participarea atât în persoană, cât și online, asigurând accesibilitatea maximă pentru toți inovatorii interesați să contribuie la viitorul explorării spațiale.

SpaceApps se adresează unui spectru larg de participanți: programatori, oameni de știință, designeri, ingineri, pasionați de tehnologie, specialiști în storytelling, producători, constructori și vizionari din întreaga lume. Acest eveniment este destinat oricui este pasionat de identificarea de soluții, indiferent de experiența anterioară în hackathoane.

\textbf{Formarea echipelor:}
\begin{itemize}[leftmargin=*]
    \item Echipele Space Apps sunt alcătuite din 2-6 membri și pot fi formate în două moduri:
    \item Participanții pot veni cu echipe deja constituite (înregistrarea se face individual)
    \item Cei care vin singuri vor fi sprijiniți de organizatori pentru a-și găsi coechipieri ideali
\end{itemize}

Evenimentul încurajează participarea studenților, inginerilor, artiștilor și experților în storytelling, deoarece provocările includ atât aspecte tehnice, cât și sarcini ce necesită abilități artistice, cunoștințe de business sau imaginație creativă.

\textbf{Condiții de participare:}
\begin{itemize}[leftmargin=*]
    \item Cunoașterea limbii engleze (prezentările și materialele sunt în engleză)
    \item Aducerea propriului laptop și a încărcătorului
    \item Entuziasm și dorința de a învăța lucruri noi
\end{itemize}

Atenție: Participanții cu vârsta sub 13 ani necesită supraveghere de către un adult pe toată durata hackathonului.

\sectiontitle{Impact și Dezvoltare Durabilă}

Proiectele realizate în cadrul SpaceApps sunt dezvoltate sub licență "Open Source", ceea ce înseamnă că pot fi folosite de oricine și pot fi dezvoltate ulterior în diferite forme. Această abordare permite ca ideile să fie transformate în startup-uri, companii sau proiecte ce pot aduce schimbări pozitive pentru mediu, umanitate și viitorul planetei.

NASA oferă cadrul, datele deschise (open-data) și temele, iar organizatorii locali caută împreună cu partenerii oameni proactivi pentru a rezolva problemele Pământului și pentru a găsi soluții pentru generațiile viitoare.

Companiile și organizațiile interesate să se alăture acestei inițiative pot contacta organizatorii pentru oportunități de sponsorizare și parteneriat la numărul de telefon 0740303764 sau prin formularele de contact disponibile pe site.

\sectiontitle{Context}

SpaceApps Challenge este un program de incubare al NASA organizat anual începând cu 2012 în hub-uri din întreaga lume. Programul a demonstrat o creștere spectaculoasă - de la primele ediții la peste 220.000 de participanți din 185 de țări și teritorii în prezent. În 2018, evenimentul a angajat peste 18.000 de persoane din 75 de țări și 200 de orașe, marcând un moment de referință în expansiunea globală a acestei inițiative.

Craiova s-a alăturat acestei rețele globale în 2018, când prima ediție locală a atras 87 de participanți care au format 12 echipe competitive. Edițiile craiovene s-au remarcat prin diversitatea participanților: elevi premianți la olimpiade de robotică și informatică, studenți de la facultățile tehnice, angajați din industria IT, antreprenori, profesori de licee și cadre universitare. Această combinație interdisciplinară a dus la soluții inovatoare și la formarea unei comunități locale active în domeniul tehnologiilor spațiale.

\sectiontitle{Înregistrare și Informații Suplimentare}

Platforma de înregistrare pentru participanți va fi disponibilă în curând la adresa:\\
\url{https://www.spaceappschallenge.org/2025/local-events/craiova/}

Evenimentul se desfășoară în conformitate cu Codul de Conduită de la Berlin, asigurând un mediu sigur și incluziv pentru toți participanții.

Pentru actualizări regulate, publicul poate urmări newsletter-ul oficial la:\\
\url{prpatrat.substack.com}

\vspace{0.5cm}
\begin{center}
\fbox{\parbox{0.9\textwidth}{
    \centering
    \textbf{Contact Media}\\
    \vspace{0.2cm}
    Pentru informații suplimentare, interviuri sau material promoțional, jurnaliștii sunt invitați să contacteze organizatorii la numărul 0740303764.
}}
\end{center}

\sectiontitle{Despre PR Pătrat și Chenist Academy}

PR Pătrat și Chenist Academy sunt organizații dedicate promovării inovației tehnologice și educației în domeniul cybersecurity și tehnologiilor spațiale, cu o experiență vastă în organizarea de evenimente internaționale de anvergură.

\vspace{1cm}
\begin{center}
    \textbf{Acest comunicat de presă este disponibil pentru publicare și distribuire liberă.}\\
    \vspace{0.3cm}
    \textbf{Pentru material media suplimentar, vă rugăm să contactați organizatorii.}\\
    \vspace{0.3cm}
    Contact: \texttt{alin@mechenici.com}
\end{center}

\end{document}
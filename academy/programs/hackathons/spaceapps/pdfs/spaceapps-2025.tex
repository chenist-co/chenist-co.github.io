\documentclass[11pt]{article}
\usepackage[a4paper, margin=1in]{geometry}
\usepackage{graphicx}
\usepackage{xcolor}
\usepackage{hyperref}
\usepackage{enumitem}
\usepackage{fontawesome}
\usepackage{tcolorbox}
\usepackage{fancyhdr}
\usepackage{tabularx}
\usepackage{booktabs}

% Define colors
\definecolor{primary}{RGB}{0, 123, 255}
\definecolor{secondary}{RGB}{108, 117, 125}
\definecolor{success}{RGB}{40, 167, 69}
\definecolor{dark}{RGB}{50, 54, 66}

% Set up hyperlinks
\hypersetup{
    colorlinks=true,
    linkcolor=primary,
    urlcolor=primary,
}

% Set up headers and footers
\pagestyle{fancy}
\fancyhf{}
\renewcommand{\headrulewidth}{0pt}
\fancyfoot[C]{\thepage}
\fancyhead[L]{\textcolor{primary}{SpaceApps Challenge 2025}}
\fancyhead[R]{\textcolor{primary}{October 4-5, 2025}}

% Custom commands
\newcommand{\sectiontitle}[1]{
    \vspace{0.3cm}
    \noindent\textcolor{primary}{\Large\textbf{#1}}
    \vspace{0.2cm}
}

\newcommand{\eventitem}[2]{
    \noindent\textbf{#1} \hfill #2\\
}

\begin{document}

\begin{center}
    \vspace*{-1cm}
    \textcolor{primary}{\Huge\textbf{PRESS RELEASE}}\\
    \vspace{0.5cm}
    \textcolor{primary}{\Large\textbf{CRAIOVA HOSTS THE 5TH EDITION OF NASA SPACEAPPS CHALLENGE 2025}}\\
    \vspace{0.3cm}
    \textcolor{secondary}{\large The world's largest annual global hackathon returns to Craiova on October 4-5, 2025}\\
    \vspace{0.5cm}
    {\normalsize Craiova, June 24, 2025}\\
    \vspace{0.8cm}
\end{center}

\begin{center}
\includegraphics[width=0.6\textwidth]{/home/alin/Documents/Stoic/dev/websites/chenist.dev/chen.ist/chenist-gh/chenist/assets/logo/spaceappschallenge.png}
\end{center}

\begin{tcolorbox}[colback=primary!10, colframe=primary, title=\textbf{Event Overview}]
Organizers PR Pătrat and Chenist Academy proudly announce that the city of Craiova will once again host the prestigious NASA SpaceApps Challenge on October 4-5, 2025. This event marks the 9th edition organized by the local team, strengthening Craiova's position on the global map of space technology innovation and continuing a successful tradition that began in 2018.
\end{tcolorbox}

\sectiontitle{About SpaceApps Challenge}

The SpaceApps Challenge is the largest annual global hackathon, engaging thousands of problem-solvers worldwide each year. The event promotes collaborative problem-solving with the aim of producing open-source solutions to address global needs applicable to both life on Earth and in space.

For 48 intensive hours, multidisciplinary teams have access to NASA's public data and its space agency partners to collaborate in developing innovative solutions to real-world challenges. The NASA Space Apps Challenge mission is to stimulate collaboration, creativity, and critical thinking, encourage interest in Earth and space exploration, increase global awareness of NASA data, and develop the next generation of scientists, engineers, technology specialists, and designers.

Since its launch in 2012, this innovative NASA incubation program has engaged over 220,000 people from more than 185 countries and territories, being administered by the Earth Science Division of the Science Mission Directorate at NASA headquarters in Washington, DC.

\sectiontitle{Program and Activities}

\textbf{Pre-Hackathon Bootcamps:}
\begin{itemize}[leftmargin=*]
    \item July 10, 2025 - location and hours to be announced
    \item August 10, 2025 - location and hours to be announced
    \item September 10, 2025 - location and hours to be announced
\end{itemize}
Bootcamps will be organized at the official "Outpost" locations of the event.

\textbf{Hackathon Schedule:}\\

\textbf{Saturday, October 4, 2025:}
\begin{itemize}[leftmargin=*]
    \item Mandatory orientation for participants
    \item Check-in, dinner, and setup of necessary software accounts
    \item Presentation of challenges and sponsors
    \item Team formation and preparation for the coding weekend
\end{itemize}

\textbf{Sunday, October 5, 2025:}
\begin{itemize}[leftmargin=*]
    \item 09:00 - Breakfast and official launch of the hackathon
    \item Project development with support from mentors and organizers
    \item 12:00 - Deadline for project submissions
    \item Afternoon - Team demonstrations (globally live-streamed)
    \item 17:00 - Winner announcements and awards ceremony
\end{itemize}

\sectiontitle{Organizers, Sponsors and Partners}

\textbf{Organizers:}
\begin{itemize}[leftmargin=*]
    \item PR Pătrat
    \item Chenist Academy
\end{itemize}

\textbf{Main Sponsor:}
\begin{itemize}[leftmargin=*]
    \item Chenist Cybersecurity
\end{itemize}

\textbf{Strategic Partners:}
\begin{itemize}[leftmargin=*]
    \item University of Craiova
    \item The Romanian-British School
    \item Chenist Digital
    \item EduTIC Cluster
    \item ARTI
    \item Roofs.Center
    \item Society for Computing Technologies
\end{itemize}

\textbf{Media Partners:}
\begin{itemize}[leftmargin=*]
    \item To be announced
\end{itemize}

\textbf{Official "Outpost" Locations:}
\begin{itemize}[leftmargin=*]
    \item To be announced
\end{itemize}

\begin{tcolorbox}[colback=secondary!10, colframe=secondary, title=\textbf{A Tradition of Excellence with Global Impact}]
SpaceApps Challenge Craiova has become a landmark event in the Romanian technological landscape since 2018, when the first local edition took place at the Oltenia Museum. In 2018, the event attracted 87 participants who formed 12 teams, demonstrating the local community's growing interest in space innovation.

Notable successes from previous editions:
\begin{itemize}
    \item In 2022, first place was won by Andrei Dumitru, an 11th-grade student from the "Ionita Asan" Theoretical High School in Caracal
    \item The 2019 edition was hosted by the Oltenia Museum and attracted approximately 100 participants, mostly with technical profiles
    \item The event has demonstrated the ability to bring together open-minded people from all fields, creating opportunities for networking and interdisciplinary collaboration
\end{itemize}

The main organizer, Alin Mechenici, brings extensive experience, being at the 9th edition of SpaceApps Challenge in 2025, after previously co-organizing the 2016 and 2017 editions at Mjardevi Science Park, Linköping, Sweden.
\end{tcolorbox}

\sectiontitle{Participation and Accessibility}

Participation in the hackathon is free, made possible by the generosity of sponsors.

The event will have a hybrid format, allowing participation both in person and online, ensuring maximum accessibility for all innovators interested in contributing to the future of space exploration.

SpaceApps addresses a wide spectrum of participants: programmers, scientists, designers, engineers, technology enthusiasts, storytelling specialists, producers, builders, and visionaries from around the world. This event is for anyone passionate about identifying solutions, regardless of previous hackathon experience.

\textbf{Team Formation:}
\begin{itemize}[leftmargin=*]
    \item Space Apps teams consist of 2-6 members and can be formed in two ways:
    \item Participants can come with already established teams (registration is done individually)
    \item Those who come alone will be supported by organizers to find ideal teammates
\end{itemize}

The event encourages the participation of students, engineers, artists, and storytelling experts, as challenges include both technical aspects and tasks requiring artistic skills, business knowledge, or creative imagination.

\textbf{Participation Requirements:}
\begin{itemize}[leftmargin=*]
    \item Knowledge of English (presentations and materials are in English)
    \item Bringing your own laptop and charger
    \item Enthusiasm and desire to learn new things
\end{itemize}

Note: Participants under 13 years of age require adult supervision throughout the hackathon.

\sectiontitle{Impact and Sustainable Development}

Projects created during SpaceApps are developed under an "Open Source" license, meaning they can be used by anyone and can be further developed in various forms. This approach allows ideas to be transformed into startups, companies, or projects that can bring positive changes for the environment, humanity, and the future of the planet.

NASA provides the framework, open data, and themes, while local organizers work with partners to find proactive people to solve Earth's problems and find solutions for future generations.

Companies and organizations interested in joining this initiative can contact the organizers for sponsorship and partnership opportunities at phone number 0740303764 or through the contact forms available on the website.

\sectiontitle{Context}

SpaceApps Challenge is a NASA incubation program organized annually since 2012 in hubs around the world. The program has shown spectacular growth - from the first editions to over 220,000 participants from 185 countries and territories today. In 2018, the event engaged over 18,000 people from 75 countries and 200 cities, marking a reference point in the global expansion of this initiative.

Craiova joined this global network in 2018, when the first local edition attracted 87 participants who formed 12 competitive teams. The Craiova editions have stood out through the diversity of participants: award-winning students in robotics and computer science olympiads, students from technical faculties, employees from the IT industry, entrepreneurs, high school teachers, and university faculty. This interdisciplinary combination has led to innovative solutions and the formation of an active local community in the field of space technologies.

\sectiontitle{Registration and Additional Information}

The registration platform for participants will soon be available at:\\
\url{https://www.spaceappschallenge.org/2025/local-events/craiova/}

The event is conducted in accordance with the Berlin Code of Conduct, ensuring a safe and inclusive environment for all participants.

For regular updates, the public can follow the official newsletter at:\\
\url{prpatrat.substack.com}

\vspace{0.5cm}
\begin{center}
\fbox{\parbox{0.9\textwidth}{
    \centering
    \textbf{Media Contact}\\
    \vspace{0.2cm}
    For additional information, interviews, or promotional material, journalists are invited to contact the organizers at 0740303764.
}}
\end{center}

\sectiontitle{About PR Pătrat and Chenist Academy}

PR Pătrat and Chenist Academy are organizations dedicated to promoting technological innovation and education in the field of cybersecurity and space technologies, with extensive experience in organizing large-scale international events.

\vspace{1cm}
\begin{center}
    \textbf{This press release is available for free publication and distribution.}\\
    \vspace{0.3cm}
    \textbf{For additional media material, please contact the organizers.}\\
    \vspace{0.3cm}
    Contact: \texttt{alin@mechenici.com}
\end{center}

\end{document}